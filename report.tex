\documentclass{jfm}

% paper sizing
\usepackage[papersize={8.5in, 11.0in}]{geometry}

% enable figures
\usepackage{graphicx}

% set path to figures
\graphicspath{{figures/}}

% enable number citations
\usepackage{natbib}

% better equations
\usepackage{amsmath}

% enable links
\usepackage{hyperref}
\hypersetup{
    colorlinks=true,
    linkcolor=[rgb]{0.86, 0.20, 0.18},  % red \ref{}
    citecolor=[rgb]{0.21, 0.44, 0.72}   % blue \cite{}
}

% symbol shortcuts
%\newcommand{\del}{\boldsymbol{\nabla}}
\newcommand{\del}{\nabla}


%--------------------------------------------------
% PREAMBLE
%--------------------------------------------------
\title[Magnetorotational instability]{Magnetorotational instability: a review}

\author[D. Larson, W. Lee, and S. Sim]{David Larson, Wonjae Lee, and Shi Sim}

\affiliation{University of California, San Diego}

\begin{document}

\maketitle


%--------------------------------------------------
% ABSTRACT
%--------------------------------------------------
\begin{abstract}
We review magnetorotational instability and its role in transporting angular 
momentum within accretion disks. Theoretical, experimental and numerical simulation
research work is surveyed.
\end{abstract}


\tableofcontents


%--------------------------------------------------
% INTRO
%--------------------------------------------------
\section{Introduction}
\label{sec:intro}

Turbulence generating magnetorotational instability (MRI) was first discovered
by \cite{Velikhov1959} and \cite{Chandrasekhar1960}, and was later
re-discovered by \cite{Balbus1998} for astrophysical applications. MRI has
since then been confirmed by robust numerical simulations, but to date has not
been verified experimentally or through observations. To understand the
importance of MRI, we have to look into the accretion disk theory where many
astrophysical phenomena take place in.

Accretion disks are disk made up of gas, dust, and plasma that rotates around
and gradually collapses onto an object in the center, e.g., leading to the
formation of a star (see Figure \ref{fig:accretion_disk} for a visualization).
However, accretion can happen only with an efficient mechanism for rapidly
transporting angular momemtum outwards \cite{Julien2010}. It was suggested that
turbulence drives angular momentum outwards but there was no known mechanism
that would generate turbulence. 


\begin{figure}[t]
    \centering
    \includegraphics[width=0.35\textwidth]{accretion_disk}
        \caption{Artist's illustration of an accretion disk in a binary system. Taken from \url{http://apod.nasa.gov/apod/ap991219.html}.}
        \label{fig:accretion_disk}
\end{figure}


These Keplerian disks satisfies the Rayleigh stability criterion
\citep[see][]{Rayleigh1916} against centrifugal instability.  Thus, there has
to be other mechanisms that generates turbulence. Many instabilities such as
barotropic  \cite{Dubrulle2005b} and finite amplitude shear instabilities
\cite{Dubrulle2005a}, \cite{Lesur2005} etc have been suggested but none of them
are sufficient for the angular momentum transport outward required
\cite{Ji2006}. Magnetic field-induced instabilities MRI seem to be the most
promising. 

MRI is a linear instability that is triggered by weak poloidal magnetic field.
It is axisymmetric and occurs in Rayleigh-stable regime where the angular
velocity decreases outwards, which is the case for the accretion we are
interested in. The efficiency of angular momentum transport depends on the
saturation of MRI \cite{Balbus1991}, \cite{Balbus1998}.

MRI is not only applicable to astrophysical phenomena but geophysical ones as
well, e.g. the Earth's magnetic field \cite{Petitdemange2008}. It is used to
better understand how planetary magnetic field might  have formed on Earth or
other planets and how the field is sustained or decayed through time.

In this report, we attempt to explain and illustrate the instability in an
astrophysical sense and go through what has been done experimentally and
numerically in terms of MRI.



%--------------------------------------------------
% BACKGROUND
%--------------------------------------------------
\section{Background}

%
% CONDUCTING FLUID
%
\subsection{Perfectly conducting fluid in magnetic field}

The magnetic field lines are tied to the conducting fluid in which they are 
embedded. We can show this by comparing a transport equation of magnetic field
with a equation for a line element moving with fluid. We first combine Ohm's 
Law in ideal conductor and Faraday's equation to get a transport equation for 
magnetic field,
\begin{align}
    \frac{\partial \mathbf{B}}{\partial t} &= \del \times (\mathbf{u}\times \mathbf{B}) \nonumber \\
    &=\mathbf{B}\cdot\del \mathbf{u} - {\mathbf{u}\cdot\del}\mathbf{B} -\mathbf{B}\del\cdot\mathbf{u}.
\end{align}

Assuming incompressibility, we get
\begin{align}
    \frac{D \mathbf{B}}{D t} = (\mathbf{B}\cdot \del) \mathbf{u},
\end{align}
where $\frac{D}{Dt}$ is the convective derivative defined as
\begin{align}
    \frac{D}{Dt} = \frac{\partial}{\partial t} + (\mathbf{u} \cdot \del)
\end{align}

Considering a short line element $d\mathbf{l}$ moving with fluid, we can express 
the rate of change of $d\mathbf{l}$ as
\begin{align}
    \frac{D}{Dt}\left(d\mathbf{l}\right) = \mathbf{u}(\mathbf{x}+d\mathbf{l})-\mathbf{u}(\mathbf{x})=(d\mathbf{l}\cdot\del)\mathbf{u}.
\end{align}
Comparing two equations above, we can conclude that $\mathbf{B}$ and 
$d\mathbf{l}$ obey the same equation \cite[see][]{Davidson2001}. Therefore, the 
field lines are frozen into the fluid.


%
% MAGNETIC TENSION
%
\subsection{Magnetic tension}

When the fluid elements are displaced from their equilibrium position, the 
magnetic field lines move together with fluid elements and behave like elastic
bands frozen-into the fluid. Using Ampere's law, the Lorenz force 
$(\mathbf{J}\times\mathbf{B})$ may be written as
\begin{align}
    \mathbf{J}\times\mathbf{B} &= - \del\left(\frac{B^2}{2\mu}\right) +\frac{(\mathbf{B}\cdot \del)\mathbf{B}}{\mu}
    =- \del\left(\frac{B^2}{2\mu}\right)+\frac{\partial}{\partial s} \left[\frac{B^2}{2\mu}\right]\hat{e}_t - \frac{B^2}{\mu R}\hat{e}_n
\end{align} 
where $R$ is radius of curvature, $s$ is a coordinate along a magnetic 
field line, and $\hat{e}_t$ and $\hat{e}_n$ are tangential and normal unit vectors 
respectively. The last two terms are magnetic tension forces in tangential and 
normal directions. These forces can also be interpreted as tensile stress of 
$B^2/2\mu$ acting on the end of the tube \cite[see][]{Davidson2001}. 

This tensile force which is also know as Faraday or Maxwell tension is 
analogous to a force acting on a spring. Considering a displacement 
$\boldsymbol{\xi}=\mathbf{v}\delta t$, the Faraday's equation can be written 
as $\delta \mathbf{B} = ikB\boldsymbol{\xi}$. The magnetic tension for small 
displacement per unit density $\rho$ is
\begin{align}
    \frac{(\mathbf{B}\cdot\del)\delta \mathbf{B}}{\mu \rho}=\frac{ikB\delta \mathbf{B}}{\mu \rho} = -\frac{k^2 B^2}{\mu\rho} \boldsymbol{\xi} = -K \boldsymbol{\xi},
\end{align}
where $K$ is comparable to the spring constant \cite[see][]{Wiki:MRI,Balbus1998}. The 
equation has same form of Hooke's Law for a spring.


%
% MRI
%
\subsection{Magnetoroational instability}
We present a relatively simple physical explanation of the magnetorotational 
instability from \cite{Balbus2011}. Consider Keplerian motion of conducting 
fluid orbiting around a central body of mass $M_c$.Two adjacent fluid elements 
$m_i$ and $m_o$, at radial position $r_i$ and $r_o$, are orbiting around 
gravitational center with angular velocities $\Omega_i=\sqrt{GM_c / r_i^{3}}$ 
and $\Omega_o=\sqrt{GM_c / r_o^{3}}$ respectively. Therefore the angular 
velocity of the inner element is higher than that of the outer elements 
$\left(d \Omega^2 / dr < 0 \right)$, but the angular momentum of the inner 
element is smaller than that of the outer element 
$\left(d r^4\Omega^2 / dr > 0 \right)$.

If a magnetic field line is connecting the conducting fluid elements, the 
magnetic field will move together with the two elements. Because of the velocity
shear in Keplerian motion, the magnetic field will be stretched and bent. 
Therefore the magnetic field will exert restoring force on the fluid element 
making the inner element pulled back and the outer element dragged forward. 
The inner element lose angular momentum, therefore it must fall to an orbit of
smaller radius. On the other hand the outer element gains angular momentum and
moves to the orbit of larger radius \cite[see][]{Balbus2011,Wiki:MRI}. This 
outward angular momentum transport makes the small initial displacement get 
larger. The magnetic fields are stretched even more and gives positive feedback
leading the system unstable. This instability is known as a magnetorotational 
instability (MRI) and is illustrated in Figure \ref{fig:mri}.

\begin{figure}[t]
    \centering
    \includegraphics[width=0.35\textwidth]{Balbus2009_diagram}
        \caption{A schematic diagram from \cite{Balbus2011} showing magnetorotational instability. Two fluid elements $m_i$ (inner) and $m_o$ (outer) orbit a mass ($M_C$) with the magnetic tension between the two elements represented by a spring. Over time $m_i$ loses angular momentum, moving closer to $M_C$ while $m_o$ gains angular momentum and moves away from $M_C$.}
        \label{fig:mri}
\end{figure}

One can show that the dispersion relation for the incompressible ideal flow 
rotating with angular velocity $\Omega_0$ is
\begin{align}
    \omega^4-\left[2\left(k^2 V_A^2\right) +\kappa^2\right]\omega^2 +\left(k^2 V_A^2\right)\left[\left(k^2 V_A^2\right)+r\frac{d \Omega^2}{dr}\right]=0,
\end{align}
where $\omega$ and $k$ are angular frequency and wave number of a perturbation
$\xi \propto e^{i(k z+\omega t)}$, $V_A = B / \sqrt{\mu \rho}$ is Alfv\'en speed
for the imposed magnetic field $B$, and $\kappa^2$ is the epicylic frequency, 
defined as
\begin{align}
    \kappa^2=4\Omega_0^2 + r \frac{d \Omega^2}{dr}
\end{align}
See \cite{Balbus1991,Balbus1998,Balbus2003} for more details. If 
$d \Omega^2 / dr < 0$, then one of $\omega^2$ has negative root for long wave 
length modes satisfying
\begin{align}
    k^2 < -\frac{r}{V_A^2}\frac{d\Omega^2}{dr}.
\end{align}

Therefore, the instability criterion for the magnetorotational instability (MRI)
is represented as radially decreasing angular velocity,
\begin{align}
    \frac{d \Omega^2}{dr}<0 \quad \text{(UNSTABLE)}
\end{align} 

Note that the MRI is different from the conventional hydrodynamic instability 
known as Couette-Taylor centrifugal instability. The Couette-Taylor centrifugal
instability criterion for axsymmetric perturbation is represented as radially 
decreasing angular momentum \cite[see][]{Charru2011}:
\begin{align}
    \frac{d r^4 \Omega^2}{dr} <0 \quad \text{(UNSTABLE)}
\end{align}

The Keplerian disk is a good example of flows which are stable to the 
hydrodynamic instability but unstable to the MRI.



%--------------------------------------------------
% THEORY
%--------------------------------------------------
\section{Theoretical Work}
\label{sec:theory}

\cite{Acheson1973} and \cite{Knobloch1992} showed the linear stability analysis
of rotating magneto-fluid bounded in coaxial cylinder. For the case of gaseous 
astrophysical disks in unbounded geometry was shown in \cite{Balbus1991}. The 
condition for stability was shown to be radially increasing angular velocity 
profile. We present a linear stability analysis of MRI for radially bounded case 
following \cite{Acheson1972}, \cite{Acheson1973a}, \cite{Knobloch1992}, and 
\cite{Julien2010}.


%
% GOVERNING EQUATIONS
%
\subsection{Governing equation}

The wave dispersion equation of a cylindrical magneto-fluid can be obtained 
from the magnetohydrodynamic (MHD) equations. Assuming the fluid is inviscid 
and perfectly conducting, the ideal MHD equations are 
\begin{align}
    \rho\left(\frac{\partial\mathbf{u}}{\partial t}+\mathbf{u}\cdot\del\mathbf{u}\right) &= -\del p +\mathbf{J}\times\mathbf{B} \\
    \frac{\partial \rho}{\partial t} + \del\cdot(\rho \mathbf{u})&=0 \\
    \frac{d}{dt}\left(\frac{p}{\rho^\gamma}\right)&=0\\
    \mathbf{E}+\mathbf{u}\times\mathbf{B}&=0 \\
    \del\times \mathbf{E} &= -\frac{\partial \mathbf{B}}{\partial t} \\
    \del \times \mathbf{B} &= \mu \mathbf{J},
\end{align}
where $\mathbf{u}$ fluid velocity, $\rho$ is fluid denstity, $p$ is pressure. 
$d/dt=\partial/\partial t +\mathbf{u}\cdot\del$ is convective derivative and 
$\gamma$ is the ratio of specific heats. $\mathbf{E}$, $\mathbf{B}$ and 
$\mathbf{J}$ are electic field, magnetic field and current density respectively
\cite[see][]{Freidberg1987}.

To investigate the magnetorotational instability, one can consider homogeneous
incompressible fluid rotating 
with angular velocity $\Omega(r)=V(r) / r$ in externally imposed magnetic 
fields $\mathbf{B}_0 = [0,B_\phi(r),B_z(r)]$. Combining electromagnetic 
equations with momentum relation, we get the appropriate MHD equations,
\begin{align}
    \frac{\partial \mathbf{u}}{\partial t}+(\mathbf{u}\cdot\del)\mathbf{u} &= -\frac{1}{\rho}\del\left(P+\frac{\mathbf{B}^2}{2\mu}\right)+\frac{1}{\mu\rho}(\mathbf{B}\cdot\del)\mathbf{B}\\
    \frac{\partial \mathbf{B}}{\partial t} +(\mathbf{u}\cdot\del)\mathbf{B} &=(\mathbf{B}\cdot\del)\mathbf{u} \\
    \del\cdot\mathbf{u}&=0\\
    \del \cdot \mathbf{B} &=0.
\end{align}


%
% LINEARIZATION
%
\subsection{Linear perturbation equation and eigenvalue problem}

We can get linearized equations by perturbing the basic state by small amount
of $\mathbf{u_1}$ and $\mathbf{b_1}$ for velocity and magnetic fields.
%\begin{align}
%\frac{\partial \mathbf{u}_1}{\partial t} +(\mathbf{u}_0\cdot \del)\mathbf{u}_1 &= -\frac{1}{\rho}\del\left(p_1+\frac{\mathbf{B}_0\cdot\mathbf{B}_1}{\mu}\right) +\frac{\mathbf{B}_0\cdot\del\mathbf{B}_1+\mathbf{B}_1\cdot\del\mathbf{B}_0}{\mu \rho}-(\mathbf{u}_1\cdot \del)\mathbf{u}_0\\
%\frac{\partial \mathbf{B}_1}{\partial t}+(\mathbf{u}_0\cdot \del)\mathbf{B}_1 &= (\mathbf{B}_0\cdot \del)\mathbf{u}_1+(\mathbf{B}_1\cdot \del)\mathbf{u}_0-(\mathbf{u}_1\cdot \del)\mathbf{B}_0\\
%\del\cdot \mathbf{u}_1&=0\\
%\del \cdot \mathbf{B}_1 &=0
%\end{align}
The linear perturbation is assumed to have the form
\begin{align}
    f=\Re\left[\hat{f}(r)e^{i(m\phi+kz+\omega t)} \right].
\end{align}

According to \cite{Acheson1972}, the normal mode equations are
\begin{align}
    \hat{b}_r &=\frac{\hat{u}_r}{\omega}\left(k B_z +\frac{m B_\phi}{r}\right) \\
    \hat{b}_\phi &= -\frac{(\hat{u}_r B_\phi)'}{i\omega} +\frac{k}{\omega}(\hat{u}_\phi B_z -\hat{u_z}B_\phi) \\
    \hat{b}_z &= -\frac{(r\hat{u}_r B_z)'}{ri\omega} - \frac{m(\hat{u_\phi} B_z - \hat{u}_z B_\phi)}{r\omega} \\
    \hat{u}_z &= -\frac{(r\hat{u}_r)'}{rik}-\frac{m\hat{u}_\phi}{rk} \\
    \left(1+\frac{m^2}{r^2k^2}\right)ri\hat{u}_\phi &= -\frac{m}{rk^2}(r\hat{u}_r)' \nonumber \\ &-\frac{\hat{u}_r}{\left(V_{Az}+\frac{m V_{A\phi}}{rk}\right)^2\frac{k^2}{\omega^2}-1}
    \left\{-\frac{2\Omega r}{\omega}+\frac{2kV_{A\phi}}{\omega^2}\left(V_{Az}+\frac{mV_{A\phi}}{rk}\right)\right\}
\end{align}
%\begin{align}
%    -i\omega\hat{b}_r&= ikB_z \hat{u}_r\\
%    -i\omega\hat{b}_\phi &= -\frac{d}{dr}(\hat{u}_r B_\phi) +ik(\hat{u}_\phi B_z -\hat{u}_z B_\phi)\\
%    -i\omega \hat{b}_z &= -\frac{1}{r}\frac{d}{dr} (r\hat{u}_r B_z)\\
%    ik\hat{u}_z &= -\frac{1}{r}\frac{d}{dr}(r \hat{u}_r)\\
%    i{(\omega^2-k^2 V_{Az}^2)}\hat{u}_\phi &= %\frac{\hat{u}_r}{r}(2\Omega r \omega +2k V_{A\phi}V_{Az})
%\end{align}
where $V_{A\phi}=B_\phi (r) / \sqrt{\mu \rho}$ and  $V_{Az}=B_z (r) / \sqrt{\mu \rho}$ 
are Alfv\'en speeds for associated external magnetic field components.

Solving the normal mode equation set for radial velocity perturbation 
$\hat{u}_r = u$ and considering axisymmetric perturbation $(m=0)$, 
\cite{Acheson1973a} obtained following eigenvalue problem,
\begin{align}
    \frac{d}{dr}\left[(\omega^2-k^2 V_{Az}^2)\left(\frac{du}{dr}+\frac{u}{r}\right)\right]-k^2\left[\omega^2-k^2 V_{Az}^2+r\frac{d}{dr}\left(\frac{V_{A\phi}^2}{r^2}-\frac{V^2}{r^2}\right)\right]u \nonumber \\
    = -\frac{4 k^2}{r^2}\frac{(k V_{A\phi} V_{Az}+\omega V)^2}{(\omega^2-k^2 V_{Az}^2)} u.
\end{align}


%
% STABILITY
%
\subsection{Stability criterion}

\subsubsection{Standard magnetorotational instability}

Consider a standard MRI of radially bounded coaxial fluid cylinder with 
externally imposed axial magnetic field but without axial current flowing. 
Therefore, we have $V_{Az} = \text{constant} \neq 0$ and 
$V_{A\phi} = 0$. Considering boundary condition 
$u(r_1)=u(r_2)=0$, we multiply the eigenvalue equation by complex conjugate 
of $u$ and integrate over radial coordinate,
\begin{align}
    (\omega^2-k^2 V_{Az}^2)^2 = \frac{k^2}{D}\int_{r_1}^{r_2}\left[\frac{\omega^2}{r^2}\frac{d}{dr}r^2V^2 -r^2 k^2 V_{Az}^2 \frac{d}{dr}\left(\frac{V^2}{r^2}\right)\right]|u|^2 dr
\end{align}   
where 
\begin{align}
    D\equiv \int_{r_1}^{r_2}\left(r \left|\frac{du}{dr}\right|^2 +\frac{|u|^2}{r}+k^2 r |u|^2 \right) dr >0
\end{align}

According to \cite{Chandrasekhar1960}, $\omega^2$ must be real. We get stable 
modes with $\omega^2>0$ and unstable modes with $\omega^2<0$. If the angular 
velocity increases radially outward, $\frac{d}{dr}\left(\frac{V^2}{r^2}\right)>0$, 
the system is stable because $\omega^2$ is bounded from below by positive number, 
\begin{align}
    \omega^2>\frac{r^2k^2 V_{Az}^2 \frac{d}{dr}\left(\frac{V^2}{r^2}\right)}{4\frac{V^2}{r}+r^2\frac{d}{dr}\left(\frac{V^2}{r^2}\right)}>0.
\end{align}
If we have radially decreasing angular velocity profile, 
$\frac{d}{dr}\left(\frac{V^2}{r^2}\right)<0$, somewhere $r_1<r<r_2$, then 
$\omega^2$ may have negative solution which makes the system unstable.

\subsubsection{Helical magnetorotational instability}

When the external nonzero magnetic fields in axial and azimuthal directions are
considered, it was found that the eigenvalue equation can be written as
\begin{align}
    \frac{d}{dr}r\frac{du}{dr}-\frac{u}{r}-k^2ru = \frac{k^2}{(\omega^2-k^2 V_{Az}^2)^2}\left[r^2 \frac{d}{dr}\left(\frac{V_{A\phi}^2-V^2}{r^2}\right)
    (\omega^2-k^2V_{Az}^2) 
    -\frac{4}{r}(kV_{A\phi}V_{Az}-\omega V)^2\right]u
\end{align}

According to \cite{Knobloch1992} and \cite{Julien2010}, the exponentially 
growing mode $\omega =-i\lambda$, $\lambda>0$ is possible when the eigenvalue 
relation has the form
\begin{align}
    (\lambda^2 +k^2 V_{Az}^2)^2 = \frac{k^2}{D}\int_{r_1}^{r_2} \left[ r^2 \frac{d}{dr}\left( \frac{V_{A\phi}^2-V^2}{r^2}\right)(\lambda^2+k^2 V_{Az}^2) + \frac{4}{r}(k V_{A\phi} V_{Az}-i\lambda V)^2 \right]|u|^2 dr.
\end{align}

Considering the imaginary part of the equation, we have 
\begin{align}
    \int_{r_1}^{r_2} \frac{1}{r}V_\phi V |u|^2 dr =0
\end{align}

\cite{Knobloch1992} showed that exponentially growing instability is only possible 
when $V_{A\phi}$ or $V$ changes sign somewhere in $r_1< r < r_2$.



%--------------------------------------------------
% EXPERIMENTS
%--------------------------------------------------
\section{Laboratory Experiments}
\label{sec:experiments}

Attempts to generate MRI in laboratory environments did not begin until 2001,
four decades after the first magnetohydrodynamic (MHD) experiments 
\citep[see][]{Donnelly1960}. In contrast to MRI, the first MHD experiments 
focused on stabilizing, rather than generating, instabilities within 
Taylor-Couette flow using magnetic fields. Experiments aimed at generating MRI
were not devised until the after published work indicating MRI's role as the
transport mechanism of angular momentum within accretion disks (need to add citation of paper).

%
% STANDARD
%
\subsection{Standard magnetorotational instability}
\label{sec:standard_mri}
Beginning with \cite{Ji2001}, researchers have attempted to produce MRI by
applying an axial magnetic field to a stable, quasi-Keplerian flow. Unfortunately,
to-date no group has been able to experimentally genereate the standard MRI due
to technical issues. Specifically, researchers have been unable to create an
experimental setup which can withstand the required flow conditions for MRI
\citep{Ji2001, Ji2002}. Additionally, researchers have had difficulty
eliminating Ekman circulation within their setups, which can transport angular
momentum and therefore whose presence can contaminate any experimental results 
\cite{Kageyama2004}.

Although definitive identification of MRI has yet to be reported, \cite{Sisan2004}
reported instabilities using a spherical Couette flow. The experimental setup
applied a magnetic field coaxially to an inner rotating sphere with an outer
fixed sphere and liquid sodium as the conducting fluid (see Figure \ref{fig:diagram_Sisan}).
Once the magnetic field was increased beyond a threshold value, instabilities
within the flow were observed which \cite{Sisan2004} classified as MRI.
\cite{Gissinger2011} later numerically reproduced the results of \cite{Sisan2004},
but identified the instabilities as shear layer instabilities, aka the 
Schercliff layer, caused by the imposed magnetic field and the spherical boundaries. 
Unfortuntately, the spherical boundaries means the Schercliff layer are unlikely
to be relevant to accretion disks which are not spherical.

\begin{figure}
    \centering
    \includegraphics[width=0.35\textwidth]{Sisan2004_diagram}
    \caption{Diagram from \cite{Sisan2004} showing the experimental setup used to generate MRI. The fixed outer steel sphere contains liquid sodium (the conducting fluid), with the magnetic field imposed coaxially to the rotating inner copper sphere. The magnetic fields and fluid velocity are measured using Hall probes and Doppler velocimetry.}
    \label{fig:diagram_Sisan}
\end{figure}



%
% HELICAL MRI
%
\subsection{Helical magnetorotational instability}

As noted in Section \ref{sec:standard_mri}, a difficulty with generation of
MRI is attaining sufficient speeds without damaging the experimental setup. 
Although researchers are actively working on solving this issue, there are also
several groups researching helical MRI (HMRI), a modified type of MRI which requires 
less extreme flow conditions. Although accretion disks do not exhibit azimuthal
magnetic fields, HMRI experiments are still valuable in understand the underlying
physics involved in MHD.

HMRI is MRI generated by applying both axial and azimuthal magnetic fields to a 
magnetized fluid, and was first proposed by \cite{Hollerbach2005}. Hollerbach's
paper showed that the addition of an azimuthal magnetic field would allow MRI
to be generated at Reynolds numbers $10^3$ less than those of standard MRI. 
Generation of HMRI has since been experimentally proven by the PROMISE experiments
\citep[see][]{Stefani2006, Stefani2007, Stefani2009, Stefani2012}. Figure 
\ref{fig:diagram_hrmi} illustrates a typical experimental setup for generating HMRI.

\begin{figure}
    \centering
    \includegraphics[width=0.35\textwidth]{HMRI_diagram}
    \caption{Diagram of a typical HMRI experimental setup from \cite{Ji2010}. The liquid metal is placed between the inner and outer cylinders, with imposed axial and azimuthal magnetic fields.}
    \label{fig:diagram_hrmi}
\end{figure}



%--------------------------------------------------
% NUMERICAL WORK
%--------------------------------------------------
\section{Numerical Work}

On top of experiments mentioned in the previous section, numerical simulations
are also employed to try to explain the anomalous viscosity and magnetic field
generation. Direct numerical simulations (DNS) are typically used with the
advance of supercomputers and dramatically decreased computing time. DNS of the
magnetohydrodynamics (MHD) Navier Stokes equations have had much success in the
determination of the existence of MRI. Many of the numerical simulations assume
axisymmetric, vertically global and incompressible flow if Alfven speed slower
than sound speed as occurs in that regime that is stable to Taylor-Couette, the
MHD equations can be simplified to be incompressible since it is not
fundamental to the instability \cite{Balbus1991} with some exceptions
\cite{Sano1998}. 

We begin with the widely used local shearing box model \cite{Balbus1991} which
has special properties of the model equations that makes it easier to simulate
(see Figure \ref{fig:shearbox}). The special properties are that the toroidal
field drops out, there is no distinction between inward and outward directions
(symmetry) and MRI becomes an exponentially growing instability, which is what
is needed \cite{Julien2010}. 

Periodic boundary conditions were used with an added shearing component to the
radial direction. ``Method of characteristics-constrained transport'' algorithm
as explained in \cite{Hawley1995} was implemented. This model with an initial
input magnetic field with zero- volume average initiates MRI but may obliterate
the magnetic field and thus suppress the turbulence. 

The comparison of the 3D simulations results from \cite{Hawley1995} with the 2D
results from \cite{Balbus1994} shows significant difference. The 2D results
have a general channel/streaming flow solution whereas the 3D results may go
through the channel flow but eventually evolves further into a turbulent flow
with some not going through the channel flow at all. Vertical component of
gravity, hence buoyancy, was ignored and since their simulations showed
significant density contrast, the shearing box model is not self-consistent.
Purely hydrodynamic turbulence did not give significant angular momentum
transport needed even when the initial conditions had strong turbulence
structures. Their 3D results suggest that accretion disks are well described by
Euler's, ignoring the viscous terms from Navier Stokes \cite{Hawley1995}.


\begin{figure}
    \begin{minipage}[t]{0.5\textwidth}
        \begin{center}
            \includegraphics[width=0.95\textwidth]{shearbox1}
        \end{center}
    \end{minipage}
    \hfill
    \begin{minipage}[t]{0.5\textwidth}
        \begin{center}
            \includegraphics[width=0.95\textwidth]{shearbox2}
        \end{center}
    \end{minipage}
    \caption{Diagram (left) and boundary conditions (right) of the shearing box model from \cite{Balbus1991}.}.
    \label{fig:shearbox}
\end{figure}


There are many variations of numerical simulations using the local shearing box
model with results worth learning from. \cite{Fromang&Papaloizou2007} and
\cite{Fromang2007} have shown that numerical results using the box model with
zero net flux are highly dependent on the numerical method used and resolution
of the grid unless explicit diffusion coefficients and the appropriate
dissipative scales are resolved . This issue has been considered in
\cite{Lesur2007} for cases with nonzero net vertical flux, where they found
that both the viscosity and resistivity affect the amount of angular momentum
transported by magnetohydrodynamic turbulence. 

\cite{Sano1998} has shown that whether saturation occurs depends on the
Elsasser number $\Delta \equiv v^2_A/\eta \Omega$, which describes the relative
balance of Lorentz forces to Coriolis forces. Non-axisymmetric numerical
simulations done by \cite{Fleming2000} suggest saturation does occur even when
Elsasser number $  \Delta \ge 1$ if non-axisymmetric disturbances are allowed
to evolve. 

Other variations of the shearing box models include vertically stratified disks
as done in \cite{Miller1999}, who found that turbulent magnetised disk can
produce a magnetised corona in laminar flow through MRI. \cite{Stone1996}
presented a 3D MHD simulation of the nonlinear evolution of MRI in a local
shearing box method with a global vertical direction and vertically stratified
disk as well. They found that the instability generated and maintained MHD
turbulence.

The use of the shearing box model has its limitations as pointed out by
\cite{Hawley1995} \cite{Regev2008} as they do not permit any dynamics involving
the background shear, which is taken as imposed and constant in space and time
\citep{Regev2008}. With regards to resolution which is tied in with the model
used, the efficiency of angular momentum transport appears to decrease with
improved resolution as shown by \cite{Fromang2007}, suggesting that transport
rates estimated on basis of ideal MHD are affected by grid-scale dissipation
and overestimate the efficiency of angular momentum extraction by MRI.
\cite{Kapyla2011} also worked on resolution dependence of $\alpha$, the
Shakura-Sunyaev number and the possible decline of the angular momentum
transport rate with decreasing Pm. Even though the shearing box model is
restrictive in resolving the behavior of the accretion disk, it is nonetheless
useful in obtaining actual evidence for MRI \cite{Balbus1998} (see Figure
\ref{fig:Balbus1998}) and understanding the dynamo process that sustains the
magnetic field in planetary systems \cite{Lesur2008}.


\begin{figure}
    \centering
    \includegraphics[width=0.50\textwidth]{Balbus1998}
    \caption{Numerical results from \cite{Balbus1998} showing the temporal evolution of MRI. The snapshots are ($R, z$) cross-sections of field lines, with the origin at the bottom left, and the time in number of orbits annotated to each snapshot.}
    \label{fig:Balbus1998}
\end{figure}


\cite{Kirillov2012} presents a unifying description of the helical and
azimuthal versions of MRI, and they also identify the universal character of
the 'Liu' limit $2(1 - 2) \approx - 0.8284$ for the critical Rossby number.
From this universal characteristics, they are led to the prediction that the
instability will be governed by a mode with an azimuthal wavenumber that is
proportional to the ratio of axial to azimuthal applied magnetic field, when
this ratio becomes large and the Rossby number is close to the Liu limit)

Apart from the local shearing box model, other 3D models have been implemented
as well. \cite{Liu2008} simulated nonlinear development of MRI in a non-ideal
magnetohydrodynamic Taylor-Couette flow (mimicking an on-going Princeton MRI
experiment) using ZEUS-MP 2.0 code (\cite{Hayes2006}), which is a
time-explicit, compressible, astrophysical ideal MHD parallel 3D code with
added viscosity and resistivity for axisymmetric flows in cylindrical
coordinates. He shows that the saturation of MRI causes a inflowing 'jet'
feature which is opposite to the usual Ekman circulation and enhances angular
momentum transport radially outward, which is what is needed, agreeing with
\cite{Hawley1995}. Stable MRI regime ($ Re \le 1600$) enhances vertical angular
momentum transport while in unstable MRI regime ($Re \ge 3200$), MRI kicks in,
resulting in more radial angular momentum transport as compared with vertical
\cite{Liu2008}.

Other general relativistic global 3D MHD codes have been implemented to show
the existence of MRI and explore other astronomical phenomena
\cite{Villiers2003},\cite{Koide2000}. \cite{Kersale2006} simulated global MRI
to look more into the nonlinear properties and the formation of a jet that
transports vertical magnetic flux. A 3D hydrodynamic simulations explores the
nonlinear evolution of vertical shear instability in accretion disk and finds
that vertical dependence destabilizes the disk, leading to velocity
fluctuations that increases angular momentum transport \cite{Arlt2004},
contrary to \cite{Hawley1995}.



%--------------------------------------------------
% CONCLUSION
%--------------------------------------------------
\section{Conclusion}
Concluding remarks



%--------------------------------------------------
% BIBLIOGRAPHY
%--------------------------------------------------
\bibliographystyle{jfm}
\bibliography{sources}


\end{document}
