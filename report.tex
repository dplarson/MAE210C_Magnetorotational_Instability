\documentclass{jfm}

% enable figures
\usepackage{graphicx}

% enable number citations
\usepackage{natbib}

\usepackage{amsmath}

% symbol shortcuts
%\newcommand{\del}{\boldsymbol{\nabla}}
\newcommand{\del}{\nabla}


%--------------------------------------------------
% PREAMBLE
%--------------------------------------------------
\title[Magnetorotational instability]{Magnetorotational instability: a review}

\author[J.~S. Sim, D.~P. Larson, and W. Lee]{Joyce Shi Sim, David P. Larson, 
    and Wonjae Lee}

\affiliation{University of California, San Diego}

\begin{document}

\maketitle


%--------------------------------------------------
% ABSTRACT
%--------------------------------------------------
\begin{abstract}
The abstract goes here like this.
\end{abstract}


%--------------------------------------------------
% INTRO
%--------------------------------------------------
\section{Introduction}
\label{sec:intro}
Background info

What is MRI? Accretion disks? Maybe provide references to explain relevance of 
MRI tobirth of stars, geophysics, etc.

Also outline what the report will cover (probably as the last paragraph).


\subsection{Example citations}
General papers: \cite{Julien2010}, \cite{Chandrasekhar1960}, \cite{Acheson1973},
\cite{Balbus1991}

Experiments: \cite{Gailitis2002}, \cite{Sisan2004}, \cite{Stefani2006},
\cite{Stefani2007}, \cite{Ji2010}, \cite{Seilmayer2012}

Numerical simulations: \cite{Kageyama2004}, \cite{Liu2008}, \cite{Gissinger2011}, 
\cite{Travnikov2011}, \cite{Kirillov2012}, \cite{Zhao2012}



%--------------------------------------------------
% PHYSICAL EXPLANATION
%--------------------------------------------------
\section{Physical Explanation}
Physical explanation of magnetohydrodynamic (MHD) equations, concept of "frozen-in-field",
analogy of magnetic and string tension, main idea of MRI



%--------------------------------------------------
% THEORY
%--------------------------------------------------
\section{Theoretical Work}
\label{sec:theory}

Derive governing equations, linearize to small perturbations, and determine 
conditions for stability.

\subsection{Governing Equations}
\cite{Julien2010} states the governing equations as

\begin{align}
    \rho \left[ \frac{\partial u}{\partial t} + (u \cdot \del) u \right] &= \del p - \frac{1}{2 \mu_0} \del B^2 + \frac{1}{\mu} (B \cdot \del) B \\
    \frac{\partial B}{\partial t} + (u \cdot \del) B &= (B \cdot \del) u \\
    \del \cdot u &= \del \cdot B = 0
\end{align}
where $\mu_0$, $B$


\subsection{Linearization}
Linearize governing equations to small perturbations


\subsection{Normal Modes}


\subsection{Stability Analysis}


\subsection{Nonlinear Analysis}



%--------------------------------------------------
% EXPERIMENTS
%--------------------------------------------------
\section{Laboratory Experiments}
Discuss on experimental work that has been done related to this topic.

Try to categorize experiments so we can group the discussion. Also, the main
focus is to explain what has been done to attempt to recreate MRI in the lab.

NOTE: most literature on MRI experiments seem to be about either 1) axial-only
magnetic fields (standard MRI) or 2) axial and azimuthal magnetic fields
(helical MRI).



%--------------------------------------------------
% NUMERICAL WORK
%--------------------------------------------------
\section{Numerical Work}
Have there been any numerical experiments done that involve the topic.

There's probably numerical simulations of both standard and helical MRI that
have already been tested in the lab, but there may also be more large-scale
simulations (e.g. planet-scale).



%--------------------------------------------------
% CONCLUSION
%--------------------------------------------------
\section{Conclusion}
Concluding remarks



%--------------------------------------------------
% BIBLIOGRAPHY
%--------------------------------------------------
\bibliographystyle{jfm}
\bibliography{sources}


\end{document}
