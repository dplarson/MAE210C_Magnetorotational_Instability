\documentclass{jfm}

% enable figures
\usepackage{graphicx}

% enable number citations
\usepackage{natbib}

\usepackage{amsmath}

% symbol shortcuts
%\newcommand{\del}{\boldsymbol{\nabla}}
\newcommand{\del}{\nabla}


%--------------------------------------------------
% PREAMBLE
%--------------------------------------------------
\title[Magnetorotational instability]{Magnetorotational instability: a review}

\author[J.~S. Sim, D.~P. Larson, and W. Lee]{Joyce Shi Sim, David P. Larson, 
    and Wonjae Lee}

\affiliation{University of California, San Diego}

\begin{document}

\maketitle


%--------------------------------------------------
% ABSTRACT
%--------------------------------------------------
\begin{abstract}
The abstract goes here like this.
\end{abstract}


%--------------------------------------------------
% INTRO
%--------------------------------------------------
\section{Introduction}
\label{sec:intro}
Background info

What is MRI? Accretion disks? Maybe provide references to explain relevance of 
MRI tobirth of stars, geophysics, etc.

Also outline what the report will cover (probably as the last paragraph).


\subsection{Example citations}
General papers: \cite{Julien2010}, \cite{Chandrasekhar1960}, \cite{Acheson1973},
\cite{Balbus1991}

Experiments: \cite{Gailitis2002}, \cite{Sisan2004}, \cite{Stefani2006},
\cite{Stefani2007}, \cite{Ji2010}, \cite{Seilmayer2012}

Numerical simulations: \cite{Kageyama2004}, \cite{Liu2008}, \cite{Gissinger2011}, 
\cite{Travnikov2011}, \cite{Kirillov2012}, \cite{Zhao2012}



%--------------------------------------------------
% PHYSICAL EXPLANATION
%--------------------------------------------------
\section{Physical Explanation}
Physical explanation of magnetohydrodynamic (MHD) equations, concept of "frozen-in-field",
analogy of magnetic and string tension, main idea of MRI



%--------------------------------------------------
% THEORY
%--------------------------------------------------
\section{Theoretical Work}
\label{sec:theory}

\cite{Acheson1973} and \cite{Knobloch1992} showed the linear stability analysis of rotating magneto-fluid bounded in coaxial cylinder. For the case of gaseous astrophysical disks in unbounded geometry was shown in \cite{Balbus1991}. The condition for stability was shown to be radially increasing angular velocity profile. We present a linear stability analysis of MRI for bounded case following \cite{Acheson1972}, \cite{Acheson1973a}, \cite{Knobloch1992} and \cite{Julien2010}.


%
% GOVERNING EQUATIONS
%
\subsection{Governing equation}

The wave dispersion equation of a cylindrical magneto-fluid can be obtained from the magnetohydrodynamic (MHD) equations. Assuming the fluid is inviscid and perfectly conducting, the ideal MHD equations are
\begin{align}
    \rho\left(\frac{\partial\mathbf{u}}{\partial t}+\mathbf{u}\cdot\del\mathbf{u}\right) &= -\del p +\mathbf{J}\times\mathbf{B} \\
    \frac{\partial \rho}{\partial t} + \del\cdot(\rho \mathbf{u})&=0 \\
    \mathbf{E}+\mathbf{u}\times\mathbf{B}&=0 \\
    \del\times \mathbf{E} &= -\frac{\partial \mathbf{B}}{\partial t} \\
    \del \times \mathbf{B} &= \mu \mathbf{J}.
\end{align}

To investigate MRI, one can consider homogeneous incompressible fluid rotating with angular velocity $\Omega(r)=\frac{V(r)}{r}$ in externally imposed magnetic fields $\mathbf{B}_0 = [0,B_\phi(r),B_z(r)]$. Combining electromagnetic equations with momentum relation, we get the appropriate MHD equations,
\begin{align}
    \frac{\partial \mathbf{u}}{\partial t}+(\mathbf{u}\cdot\del)\mathbf{u} &= -\frac{1}{\rho}\del\left(P+\frac{\mathbf{B}^2}{2\mu}\right)+\frac{1}{\mu\rho}(\mathbf{B}\cdot\del)\mathbf{B}\\
    \frac{\partial \mathbf{B}}{\partial t} +(\mathbf{u}\cdot\del)\mathbf{B} &=(\mathbf{B}\cdot\del)\mathbf{u} \\
    \del\cdot\mathbf{u}&=0\\
    \del \cdot \mathbf{B} &=0
\end{align}


%
% LINEARIZATION
%
\subsection{Linear perturbation equation and eigenvalue problem}

We can get linearized equations by perturbing the basic state by small amount of $\mathbf{u_1}$ and $\mathbf{b_1}$ for velocity and magnetic fields.
%\begin{align}
%\frac{\partial \mathbf{u}_1}{\partial t} +(\mathbf{u}_0\cdot \del)\mathbf{u}_1 &= -\frac{1}{\rho}\del\left(p_1+\frac{\mathbf{B}_0\cdot\mathbf{B}_1}{\mu}\right) +\frac{\mathbf{B}_0\cdot\del\mathbf{B}_1+\mathbf{B}_1\cdot\del\mathbf{B}_0}{\mu \rho}-(\mathbf{u}_1\cdot \del)\mathbf{u}_0\\
%\frac{\partial \mathbf{B}_1}{\partial t}+(\mathbf{u}_0\cdot \del)\mathbf{B}_1 &= (\mathbf{B}_0\cdot \del)\mathbf{u}_1+(\mathbf{B}_1\cdot \del)\mathbf{u}_0-(\mathbf{u}_1\cdot \del)\mathbf{B}_0\\
%\del\cdot \mathbf{u}_1&=0\\
%\del \cdot \mathbf{B}_1 &=0
%\end{align}
The axisymmetric perturbation is assumed to have the form
\begin{align}
    \psi=\Re\left[\hat{\psi}(r)e^{i(kz-\omega t)} \right].
    \end{align}
    Accoring to \cite{Acheson1972}, the normal mode equations are
    \begin{align}
    -i\omega\hat{b}_r&= ikB_z \hat{u}_r\\
    -i\omega\hat{b}_\phi &= -\frac{d}{dr}(\hat{u}_r B_\phi) +ik(\hat{u}_\phi B_z -\hat{u}_z B_\phi)\\
    -i\omega \hat{b}_z &= -\frac{1}{r}\frac{d}{dr} (r\hat{u}_r B_z)\\
    ik\hat{u}_z &= -\frac{1}{r}\frac{d}{dr}(r \hat{u}_r)\\
    i{(\omega^2-k^2 V_{Az}^2)}\hat{u}_\phi &= \frac{\hat{u}_r}{r}(2\Omega r \omega +2k V_{A\phi}V_{Az})
\end{align}
where $V_{A\phi}=\frac{B_\phi (r)}{\sqrt{\mu\rho}}$, $V_{Az}=\frac{B_z (r)}{\sqrt{\mu\rho}}$ are Alfv\'en speeds for associated external magnetic field components.

Solving the normal mode equation set for radial velocity perturbation $\hat{u}_r = u$, \cite{Acheson1973a} obtained following eigenvalue problem,
\begin{align}
\frac{d}{dr}\left[(\omega^2-k^2 V_{Az}^2)\left(\frac{du}{dr}+\frac{u}{r}\right)\right]-k^2\left[\omega^2-k^2 V_{Az}^2+r\frac{d}{dr}\left(\frac{V_{A\phi}^2}{r^2}-\frac{V^2}{r^2}\right)\right]u \nonumber \\
= -\frac{4 k^2}{r^2}\frac{(k V_{A\phi} V_{Az}+\omega V)^2}{(\omega^2-k^2 V_{Az}^2)} u.
\end{align}


%
% STABILITY
%
\subsection{Stability criterion}

\subsubsection{Standard magnetorotational instability}

Consider a standard MRI of radially bounded coaxial fluid cylinder with externally imposed axial magnetic field but without axial current flowing. Therefore, we have $V_{Az}=\frac{B_z}{\sqrt{\rho\mu}}=const \neq 0$ and $V_{A\phi}=\frac{B_\phi}{\sqrt{\rho\mu}}=0$. Considering boundary condition $u(r_1)=u(r_2)=0$, we multiply the eigenvalue equation by complex conjugate of $u$ and integrate over radial coordinate,
\begin{align}
    (\omega^2-k^2 V_{Az}^2)^2 = \frac{k^2}{D}\int_{r_1}^{r_2}\left[\frac{\omega^2}{r^2}\frac{d}{dr}r^2V^2 -r^2 k^2 V_{Az}^2 \frac{d}{dr}\left(\frac{V^2}{r^2}\right)\right]|u|^2 dr
\end{align}   
where 
\begin{align}
    D\equiv \int_{r_1}^{r_2}\left(r \left|\frac{du}{dr}\right|^2 +\frac{|u|^2}{r}+k^2 r |u|^2 \right) dr >0
\end{align}

Accoring to \cite{Chandrasekhar1960}, $\omega^2$ must be real. We get stable modes with $\omega^2>0$ and unstable modes with $\omega^2<0$. If the angular velocity increases radially outward, $\frac{d}{dr}\left(\frac{V^2}{r^2}\right)>0$, the system is stable because $\omega^2$ is bounded from below by positive number, 
\begin{align}
    \omega^2>\frac{r^2k^2 V_{Az}^2 \frac{d}{dr}\left(\frac{V^2}{r^2}\right)}{4\frac{V^2}{r}+r^2\frac{d}{dr}\left(\frac{V^2}{r^2}\right)}>0.
\end{align}
If we have radially decreasing anguar velocity profile, $\frac{d}{dr}\left(\frac{V^2}{r^2}\right)<0$, somewhere $r_1<r<r_2$, then $\omega^2$ may have negative solution which makes the system unstable.

\subsubsection{Helical magnetorotational instability}

When the external nonzero magnetic fields in axial and azimuthal directions are considered, it was found that the eigenvalue equation can be written as
\begin{align}
    \frac{d}{dr}r\frac{du}{dr}-\frac{u}{r}-k^2ru = \frac{k^2}{(\omega^2-k^2 V_{Az}^2)^2}\left[r^2 \frac{d}{dr}\left(\frac{V_{A\phi}^2-V^2}{r^2}\right)
    (\omega^2-k^2V_{Az}^2) 
    -\frac{4}{r}(kV_{A\phi}V_{Az}-\omega V)^2\right]u
\end{align}

According to \cite{Knobloch1992} and \cite{Julien2010}, the exponentially growing mode $\omega =-i\lambda$, $\lambda>0$ is possilbe when the eigenvalue relation has following form
\begin{align}
    (\lambda^2 +k^2 V_{Az}^2)^2 = \frac{k^2}{D}\int_{r_1}^{r_2} \left[ r^2 \frac{d}{dr}\left( \frac{V_{A\phi}^2-V^2}{r^2}\right)(\lambda^2+k^2 V_{Az}^2) + \frac{4}{r}(k V_{A\phi} V_{Az}-i\lambda V)^2 \right]|u|^2 dr.
\end{align}

Considering the imaginary part of the equation, we have 
\begin{align}
    \int_{r_1}^{r_2} \frac{1}{r}V_\phi V |u|^2 dr =0
\end{align}

\cite{Knobloch1992} shows exponentially growing instability is only possible when $V_{A\phi}$ or $V$ changes sign somewhere in $r_1< r < r_2$.



%--------------------------------------------------
% EXPERIMENTS
%--------------------------------------------------
\section{Laboratory Experiments}
Discuss on experimental work that has been done related to this topic.

Try to categorize experiments so we can group the discussion. Also, the main
focus is to explain what has been done to attempt to recreate MRI in the lab.

NOTE: most literature on MRI experiments seem to be about either 1) axial-only
magnetic fields (standard MRI) or 2) axial and azimuthal magnetic fields
(helical MRI).



%--------------------------------------------------
% NUMERICAL WORK
%--------------------------------------------------
\section{Numerical Work}
Have there been any numerical experiments done that involve the topic.

There's probably numerical simulations of both standard and helical MRI that
have already been tested in the lab, but there may also be more large-scale
simulations (e.g. planet-scale).



%--------------------------------------------------
% CONCLUSION
%--------------------------------------------------
\section{Conclusion}
Concluding remarks



%--------------------------------------------------
% BIBLIOGRAPHY
%--------------------------------------------------
\bibliographystyle{jfm}
\bibliography{sources}


\end{document}
